\section{Budowanie aplikacji przy użyciu Mavena i Dockera}

W projekcie wykorzystano zarówno Maven, jak i Docker do zarządzania procesem budowy aplikacji. Oba narzędzia pełnią różne role: Maven służy do zarządzania zależnościami, kompilacji i testowaniem aplikacji, natomiast Docker umożliwia tworzenie kontenerów dla aplikacji oraz jej środowiska. Poniżej przedstawiono szczegóły dotyczące roli, jaką odgrywają oba narzędzia w procesie budowy aplikacji.

\subsection{Budowanie aplikacji przy użyciu Mavena}

\begin{itemize}
    \item \textbf{Zarządzanie zależnościami}: Maven automatycznie pobiera i zarządza zależnościami z repozytoriów, takimi jak Maven Central, co umożliwia łatwe dodawanie bibliotek i frameworków do projektu (np. Spring Boot, JPA, itp.).
    \item \textbf{Kompilacja kodu}: Maven kompiluje aplikację Java, wykorzystując plik \texttt{pom.xml}, w którym określane są wszystkie niezbędne zależności oraz ustawienia kompilacji.
    \item \textbf{Uruchamianie testów jednostkowych}: Maven jest odpowiedzialny za uruchamianie testów jednostkowych, takich jak te napisane przy użyciu JUnit. Pomaga to w zapewnieniu jakości aplikacji poprzez automatyczne wykonywanie testów przed jej wdrożeniem.
    \item \textbf{Tworzenie plików JAR/WAR}: Maven generuje pliki JAR (Java ARchive) lub WAR (Web ARchive), które zawierają skompilowany kod aplikacji oraz wszystkie zależności, co pozwala na łatwe wdrożenie aplikacji w środowisku produkcyjnym.
    \item \textbf{Pluginy do budowy aplikacji}: Maven wykorzystuje różne pluginy, takie jak \texttt{maven-compiler-plugin} do kompilacji kodu, \texttt{maven-surefire-plugin} do uruchamiania testów, czy \texttt{maven-assembly-plugin} do pakowania aplikacji w finalny artefakt.
\end{itemize}

Przykład polecenia do budowy aplikacji przy użyciu Mavena:

\begin{verbatim}
mvn clean install
\end{verbatim}

\subsection{Budowanie aplikacji przy użyciu Dockera}

Docker jest narzędziem do tworzenia kontenerów, które umożliwia uruchomienie aplikacji w izolowanym środowisku, co zapewnia jej przenośność między różnymi środowiskami. W projekcie Docker pełni rolę w procesie budowy i uruchamiania aplikacji w kontenerach.

\begin{itemize}
    \item \textbf{Tworzenie obrazów Docker}: Docker wykorzystuje plik \texttt{Dockerfile} do budowy obrazu kontenera. Obraz ten zawiera wszystkie niezbędne zależności oraz środowisko do uruchomienia aplikacji. W przypadku aplikacji składającej się z kilku usług, jak frontend, backend i Node.js (do integracji z Solaną), każdy z tych komponentów ma własny \texttt{Dockerfile}, który definiuje sposób budowy obrazu dla danej części aplikacji. Po stworzeniu obrazów, kontenery mogą zostać uruchomione razem za pomocą pliku \texttt{docker-compose.yml}, który zarządza wszystkimi usługami w jednej sieci.
    
    \item \textbf{Kompilacja aplikacji w kontenerze}: W niektórych przypadkach aplikacja może być kompilowana bezpośrednio w kontenerze Docker. Na przykład kontener backendowy, w którym znajduje się aplikacja Spring Boot, uruchamia proces kompilacji przy użyciu Mavena, a następnie uruchamia aplikację w tym samym kontenerze. Dzięki temu cały proces budowy aplikacji i jej uruchamiania odbywa się w izolowanym środowisku, bez konieczności instalowania Mavena lub JDK na systemie operacyjnym.
    
    \item \textbf{Uruchamianie aplikacji w kontenerze}: Po zbudowaniu obrazu Docker, kontener jest uruchamiany, co oznacza, że aplikacja działa w izolowanym środowisku. Docker pozwala na uruchomienie aplikacji zarówno w środowisku developerskim, jak i produkcyjnym, z zachowaniem pełnej przenośności. W przypadku aplikacji wieloskładnikowej, jak frontend, backend oraz usługa Node.js do komunikacji z blockchainem Solany, wszystkie kontenery komunikują się między sobą poprzez wspólną sieć Docker (\texttt{my\_network}), co zapewnia łatwą wymianę danych.
    
    \item \textbf{Zarządzanie zależnościami środowiska}: Docker umożliwia zarządzanie wszystkimi zależnościami aplikacji w ramach kontenera. Na przykład, kontener z bazą danych PostgreSQL jest uruchamiany w osobnym kontenerze, zapewniając pełną izolację od aplikacji backendowej. Wspólny plik \texttt{docker-compose.yml} definiuje, jak kontenery frontendowy, backendowy oraz Node.js (do Solany) mają współdziałać w tej samej sieci, co zapewnia łatwe zarządzanie konfiguracją środowiska.
\end{itemize}


Przykład polecenia do budowy obrazu Docker:

\begin{verbatim}
docker compose up --build
\end{verbatim}

To polecenie buduje obraz Docker od nowa na podstawie instrukcji zawartych w różnych plikach \texttt{Dockerfile}, a następnie pozwala na uruchomienie aplikacji w kontenerach.

\subsection{Integracja Mavena i Dockera}

W projekcie wykorzystano zarówno Maven, jak i Docker, aby połączyć zalety obu narzędzi. Maven służy do kompilacji i zarządzania zależnościami aplikacji, podczas gdy Docker zapewnia izolowane i przenośne środowisko uruchomieniowe. Oba narzędzia współpracują ze sobą, umożliwiając pełną automatyzację procesu budowy aplikacji oraz jej uruchamianie w różnych środowiskach, zarówno developerskich, jak i produkcyjnych.

Na przykład, Docker może być użyty do uruchamiania kontenera z aplikacją, która została wcześniej zbudowana za pomocą Mavena. Docker umożliwia łatwe zarządzanie wersjami aplikacji oraz zapewnia jej spójność niezależnie od środowiska, w którym jest uruchamiana.