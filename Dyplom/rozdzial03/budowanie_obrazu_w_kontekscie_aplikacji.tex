\subsection{Budowanie obrazu Docker w kontekście aplikacji}

W kontekście Twojej aplikacji, proces budowy obrazów Docker jest kluczowy do zapewnienia spójności środowiska, w którym aplikacja jest uruchamiana. Dzięki wykorzystaniu Docker, zarówno frontend, jak i backend mogą być uruchamiane w odizolowanych środowiskach, co zapewnia ich przenośność oraz łatwość konfiguracji. Poniżej opisano szczegóły procesu budowy obrazów Docker dla obu części aplikacji.

\subsubsection{Budowanie obrazu Docker dla aplikacji backendowej}

Dla aplikacji backendowej, która jest oparta na Javie i wykorzystuje Spring Boot, plik \texttt{Dockerfile} będzie zawierał następujące instrukcje:

\begin{verbatim}
FROM openjdk:21-jdk-slim

# Ustawienie katalogu roboczego
WORKDIR /app

# Kopiowanie pliku JAR aplikacji do kontenera
COPY target/myapp.jar /app/myapp.jar

# Określenie komendy uruchamiającej aplikację
CMD ["java", "-jar", "/app/myapp.jar"]
\end{verbatim}

1. \texttt{FROM openjdk:21-jdk-slim} – użycie obrazu bazowego Java 21, zoptymalizowanego pod względem rozmiaru.
2. \texttt{WORKDIR /app} – ustawia katalog roboczy, w którym będą znajdować się pliki aplikacji.
3. \texttt{COPY target/myapp.jar /app/myapp.jar} – kopiuje plik JAR aplikacji do kontenera.
4. \texttt{CMD ["java", "\texttt{-}jar", "/app/myapp.jar"]} – uruchamia aplikację Java w kontenerze.

W celu zbudowania obrazu dla aplikacji backendowej, wystarczy wykonać polecenie:

\begin{verbatim}
docker build -t backend-app .
\end{verbatim}

Polecenie to buduje obraz Docker, używając pliku \texttt{Dockerfile} w bieżącym katalogu, a tag \texttt{backend-app} przypisuje nazwę obrazowi.

\subsubsection{Budowanie obrazu Docker dla aplikacji frontendowej}

Dla aplikacji frontendowej, która jest oparta na React, proces budowy obrazu Docker będzie wyglądał nieco inaczej. W przypadku aplikacji React, obraz będzie zawierał środowisko Node.js oraz instrukcje do budowy aplikacji, a także do jej uruchomienia. Przykładowy plik \texttt{Dockerfile} wygląda następująco:

\begin{verbatim}
FROM node:18-slim

# Ustawienie katalogu roboczego
WORKDIR /app

# Kopiowanie plików projektu do kontenera
COPY package.json package-lock.json /app/
RUN npm install

# Kopiowanie pozostałych plików aplikacji
COPY . /app/

# Budowanie aplikacji
RUN npm run build

# Określenie portu, na którym będzie działać aplikacja
EXPOSE 3000

# Uruchamianie aplikacji
CMD ["npm", "start"]
\end{verbatim}

1. \texttt{FROM node:18-slim} – użycie obrazu bazowego Node.js w wersji 18.
2. \texttt{WORKDIR /app} – ustawia katalog roboczy w kontenerze.
3. \texttt{COPY package.json package-lock.json /app/} – kopiuje pliki konfiguracyjne npm do kontenera.
4. \texttt{RUN npm install} – instaluje zależności aplikacji.
5. \texttt{COPY . /app/} – kopiuje wszystkie pozostałe pliki aplikacji do kontenera.
6. \texttt{RUN npm run build} – buduje aplikację React.
7. \texttt{EXPOSE 3000} – otwiera port 3000 dla aplikacji frontendowej.
8. \texttt{CMD ["npm", "start"]} – uruchamia aplikację React w kontenerze.

Aby zbudować obraz dla aplikacji frontendowej, należy wykonać polecenie:

\begin{verbatim}
docker build -t frontend-app .
\end{verbatim}

To polecenie buduje obraz Docker, używając pliku \texttt{Dockerfile} i taguje go nazwą \texttt{frontend-app}.

\subsubsection{Integracja z Docker Compose}

W celu ułatwienia zarządzania i uruchamiania zarówno aplikacji frontendowej, jak i backendowej w środowisku Docker, można użyć Docker Compose, narzędzia do orkiestracji kontenerów. Plik \texttt{docker-compose.yml} może wyglądać następująco:

\begin{verbatim}
version: '3'
services:
  backend:
    build:
      context: ./backend
    ports:
      - "8080:8080"
  frontend:
    build:
      context: ./frontend
    ports:
      - "3000:3000"
    depends_on:
      - backend
\end{verbatim}

1. W sekcji \texttt{services} zdefiniowane są dwa serwisy: \texttt{backend} i \texttt{frontend}.
2. \texttt{build} określa, gdzie znajduje się plik \texttt{Dockerfile} dla każdego serwisu (odpowiednio \texttt{./backend} i \texttt{./frontend}).
3. \texttt{ports} mapuje porty z kontenera na porty hosta, aby aplikacje były dostępne z zewnątrz.
4. \texttt{depends\_on} zapewnia, że kontener frontendowy będzie uruchamiany po backendzie.

Aby uruchomić oba serwisy w jednym poleceniu, wystarczy wykonać:

\begin{verbatim}
docker-compose up --build
\end{verbatim}

Polecenie to buduje obrazy kontenerów i uruchamia aplikacje backendową i frontendową w oddzielnych kontenerach, z odpowiednimi zależnościami.

\subsection{Podsumowanie}

Budowanie obrazów Docker dla aplikacji frontendowej i backendowej w Twoim projekcie jest kluczowym krokiem w zapewnieniu przenośności i izolacji środowisk. Docker umożliwia łatwe zarządzanie zależnościami oraz uruchamianie aplikacji w odizolowanych środowiskach, co upraszcza proces deploymentu zarówno w środowisku developerskim, jak i produkcyjnym. Integracja z Docker Compose pozwala na zarządzanie wieloma kontenerami w sposób spójny i wydajny.