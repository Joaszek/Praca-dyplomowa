\section{Bezpieczeństwo aplikacji}

Aplikacja wykorzystuje framework \texttt{Spring Security} w celu zapewnienia bezpieczeństwa i kontrolowania dostępu do zasobów. Dzięki zastosowaniu różnych mechanizmów uwierzytelniania i autoryzacji, takich jak \texttt{JWT} (JSON Web Tokens), oraz \texttt{Basic Authentication}, aplikacja jest chroniona przed nieautoryzowanym dostępem. Poniżej przedstawiono kluczowe komponenty zapewniające bezpieczeństwo aplikacji.

\subsection{Konfiguracja bezpieczeństwa} (\texttt{SecurityConfig.java})

Klasa \texttt{SecurityConfig} zawiera pełną konfigurację bezpieczeństwa aplikacji, w tym ustawienia filtrów bezpieczeństwa, które kontrolują dostęp do zasobów. Endpotiny \texttt{/client/register} i \texttt{/client/login}, służą użytkownikowi do rejestracji lub logowania, tak aby mógł mieć dostęp do platformy. 

Klasa ta zawiera również konfigurację \texttt{CORS}, która zezwala na dostęp do zasobów aplikacji z określonych domen (np. \texttt{http://localhost:3000}). Dzięki temu aplikacja ma możliwość komunikacji z frontendem.

\subsection{Generowanie i weryfikacja JWT} (\texttt{JwtTokenUtils.java})

Klasa \texttt{JwtTokenUtils} odpowiedzialna jest za generowanie oraz weryfikację tokenów \texttt{JWT}. Tokeny te są wykorzystywane do uwierzytelniania użytkowników i zapewnienia im dostępu zasobów aplikacji. Klasa ta wykonuje następujące operacje:
\begin{itemize}
    \item Generowanie tokenów \texttt{JWT} z unikalnymi danymi użytkownika.
    \item Weryfikacja ważności tokenu, w tym sprawdzanie daty wygaśnięcia.
    \item Ustalenie, czy użytkownik w tokenie jest tym samym, który jest zapisany w bazie danych.
\end{itemize}
Po weryfikacji tokenu, użytkownik zostaje uwierzytelniony i przypisany do kontekstu bezpieczeństwa aplikacji.

\subsection{Filtrowanie tokenów JWT} (\texttt{JwtAccessTokenFilter.java})

Klasa \texttt{JwtAccessTokenFilter} jest odpowiedzialna za filtrację przychodzących żądań HTTP. Filtr ten sprawdza nagłówek \texttt{Authorization} w żądaniach, aby wyodrębnić i zweryfikować token \texttt{JWT}. Jeśli token jest ważny, uzyskuje dostęp do platformy. W przypadku, gdy token jest nieważny lub nie istnieje, aplikacja zwraca odpowiedź o statusie \texttt{UNAUTHORIZED}.

\subsection{Zarządzanie danymi użytkowników} (\texttt{UserInfoManager.java})

Klasa \texttt{UserInfoManager} implementuje interfejs \texttt{UserDetailsService} i jest odpowiedzialna za załadowanie użytkownika z bazy danych na podstawie nazwy użytkownika. Metoda \texttt{loadUserByUsername} korzysta z repozytorium \texttt{ClientRepository}, aby odnaleźć użytkownika, a następnie dostarcza jego dane do kontekstu bezpieczeństwa aplikacji, co pozwala na uwierzytelnianie.

\subsection{Rekordy RSA} (\texttt{RSAKeyRecord.java})

Klasa \texttt{RSAKeyRecord} przechowuje klucze RSA (publiczny i prywatny), które są wykorzystywane do weryfikacji i generowania tokenów \texttt{JWT}. Klucz publiczny jest używany do weryfikacji podpisu tokenu, natomiast klucz prywatny służy do jego generowania. Jest to istotny element systemu autoryzacji, zapewniający integralność i bezpieczeństwo transmisji danych.

\subsection{Podstawowa autentykacja} (\texttt{BasicAuthenticationSecurityConfiguration.java})

Klasa \texttt{BasicAuthenticationSecurityConfiguration} wprowadza mechanizm podstawowej autentykacji. Choć nie zawiera jeszcze pełnej logiki konfiguracyjnej, jest przygotowana do użycia w przyszłości w celu rozszerzenia mechanizmów uwierzytelniania w aplikacji.