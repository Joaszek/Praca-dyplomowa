\subsection{Podział klas i wzorców projektowych w aplikacji}

Aplikacja jest podzielona na różne warstwy, każda z nich odpowiedzialna za określoną funkcjonalność. Wzorce projektowe takie jak \texttt{DTO}, \texttt{Service Layer}, \texttt{Repository}, oraz \texttt{Controller} zostały zastosowane, aby zapewnić czystą architekturę i separację odpowiedzialności.

\subsubsection{Podział klas}

W aplikacji występują trzy główne warstwy: \texttt{Model}, \texttt{Service} i \texttt{Controller}, z odpowiednimi klasami, które pełnią określoną rolę.

\begin{lstlisting}[language=Java, caption={Podział klas w aplikacji}]
Model:
- SolanaAccount, SolanaTransaction, EthereumBlock
  Reprezentują dane związane z blockchainami, głównie do przenoszenia danych.
- Cryptocurrency, Categories, GlobalMarket, NFT
  Reprezentują dane kryptowalutowe i NFT, kluczowe dla funkcji aplikacji.

Service:
- BlockchainService, CryptocurrencyService, NFTService
  Zarządzają logiką biznesową, przetwarzając i udostępniając dane.

Controller:
- BlockchainController, CryptocurrencyController, NFTController
  Obsługują zapytania HTTP, przekazując je do serwisów i zwracając odpowiedzi.
\end{lstlisting}


\subsubsection{Wzorce projektowe}

W aplikacji zastosowano kilka popularnych wzorców projektowych, które pomagają w organizacji kodu oraz umożliwiają łatwą jego rozbudowę i konserwację.

\begin{lstlisting}[language=Java, caption={Wzorce projektowe w aplikacji}]
DTO (Data Transfer Object):
- Używany do przenoszenia danych między warstwami aplikacji w ujednoliconej formie.
- Przykłady: AccountDTO, TransactionDTO, BlockDTO
  Transferują dane między backendem a frontendem.

Service Layer:
- Realizuje logikę biznesową, zapewnia modularność aplikacji.
- Przykłady: BlockchainService, CryptocurrencyService, NFTService
  Zarządzają operacjami na blockchainach, kryptowalutach i NFT.

Repository:
- Oddziela logikę dostępu do danych od reszty aplikacji.
- Przykłady: CategoryRepository, ClientRepository, NFTRepository
  Umożliwiają interakcję z bazą danych w przejrzysty sposób.

Singleton:
- Zapewnia, że dany obiekt jest tworzony tylko raz i współdzielony.
\end{lstlisting}


\subsubsection{Kontrolery i Repozytoria}

\begin{lstlisting}[language=Java, caption={Kontrolery i Repozytoria w aplikacji}]
Kontrolery:
- EthereumController
  Obsługuje zapytania związane z blockchainem Ethereum.
- CryptocurrencyController
  Obsługuje zapytania dotyczące danych kryptowalutowych.
- NFTController
  Zarządza operacjami związanymi z NFT.

Repozytoria:
- ClientRepository
  Odpowiada za dostęp do danych konta użytkownika.
- CryptocurrencyRepository
  Zarządza danymi kryptowalut.
- CategoryRepository
  Obsługuje dane związane z kategoriami.
\end{lstlisting}
