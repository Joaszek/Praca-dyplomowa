\section{Budowanie Aplikacji za pomocą Maven i Docker}

Projekt wykorzystuje zarówno Maven, jak i Docker do zarządzania procesem budowy aplikacji. Narzędzia te pełnią różne role: Maven obsługuje zarządzanie zależnościami, kompilację i testowanie aplikacji, natomiast Docker umożliwia konteneryzację aplikacji i jej środowiska. Poniżej znajduje się szczegółowe wyjaśnienie ról tych narzędzi w procesie budowy.

\subsection{Maven}

Maven to narzędzie do zarządzania projektami i automatyzacji budowy, szeroko stosowane w ekosystemie Java. Wspiera cały cykl życia aplikacji, od kompilacji i testowania po wdrażanie. Maven upraszcza zarządzanie zależnościami, automatyzuje pobieranie bibliotek, kontrolę wersji i skutecznie organizuje projekty.

\subsubsection{Zalety Użycia Maven}

\begin{itemize}
    \item \textbf{Zarządzanie Zależnościami:} Maven automatycznie pobiera i zarządza zależnościami z centralnego repozytorium Maven lub innych źródeł, eliminując konieczność ręcznego pobierania i wersjonowania.
    \item \textbf{Uproszczony Proces Budowy:} Maven automatyzuje proces kompilacji, testowania i pakowania. Standardowa struktura katalogów projektu zapewnia łatwość utrzymania i skalowalność.
    \item \textbf{Rozbudowane Wsparcie dla Wtyczek:} Maven obsługuje liczne wtyczki do automatyzacji zadań, takich jak uruchamianie testów jednostkowych, generowanie raportów, tworzenie dokumentacji oraz budowanie aplikacji w różnych formatach (np. JAR, WAR).
    \item \textbf{Integracja z CI/CD:} Maven bezproblemowo integruje się z narzędziami CI/CD, takimi jak Jenkins czy GitLab CI, umożliwiając automatyczne budowanie, testowanie i wdrażanie w różnych środowiskach.
    \item \textbf{Standaryzowany Proces Budowy:} Proces budowy jest zdefiniowany w pliku \texttt{pom.xml}, który centralizuje wszystkie informacje o zależnościach, wtyczkach i konfiguracji budowy.
\end{itemize}

\subsubsection{Proces Budowy Aplikacji za pomocą Maven}

Maven wspiera wiele zadań w cyklu życia aplikacji, w tym:

\begin{itemize}
    \item \textbf{Kompilacja:} Kompiluje kod źródłowy w celu generowania artefaktów, takich jak pliki JAR lub WAR.
    \item \textbf{Testowanie:} Uruchamia testy jednostkowe (np. z użyciem JUnit), aby zapewnić jakość aplikacji.
    \item \textbf{Pakowanie:} Pakuje skompilowany kod w formatach nadających się do wdrożenia (np. JAR, WAR).
    \item \textbf{Instalacja:} Instaluje zbudowane artefakty w lokalnym repozytorium w celu ponownego użycia w innych projektach.
\end{itemize}

\subsubsection{Plik Konfiguracyjny \texttt{pom.xml}}

Podstawą projektu Maven jest plik \texttt{pom.xml}, który definiuje wszystkie konfiguracje projektu:

\begin{itemize}
    \item \textbf{Zależności:} Lista bibliotek używanych w projekcie, które Maven pobiera automatycznie.
    \item \textbf{Wtyczki:} Specyfikacja wtyczek do różnych zadań, takich jak kompilacja, testowanie i wdrażanie.
    \item \textbf{Profile:} Konfiguracje budowy dostosowane do różnych środowisk (np. produkcja, testowanie).
\end{itemize}

\subsubsection{Budowanie Aplikacji za pomocą Maven}

Maven automatyzuje etapy cyklu życia aplikacji, takie jak \texttt{clean}, \texttt{compile}, \texttt{test}, \texttt{package}, \texttt{install} i \texttt{deploy}. Typowe polecenie budowy Maven to:

\begin{verbatim}
mvn clean install
\end{verbatim}

Polecenie to usuwa poprzednie wyniki budowy, kompiluje aplikację, uruchamia testy jednostkowe i tworzy finalny artefakt (np. plik JAR).

\subsection{Docker}

Docker to narzędzie do konteneryzacji, które tworzy izolowane środowiska do uruchamiania aplikacji, zapewniając ich przenośność na różnych systemach. W tym projekcie Docker jest używany do budowy i uruchamiania aplikacji w kontenerach.

\subsubsection{Zalety Użycia Dockera}

\begin{itemize}
    \item \textbf{Izolacja Środowiska:} Zapewnia spójne działanie aplikacji niezależnie od systemu operacyjnego.
    \item \textbf{Przenośność:} Kontenery Docker mogą działać na dowolnej platformie (Windows, Linux, macOS).
    \item \textbf{Uproszczone Zarządzanie Zależnościami:} Wszystkie zależności aplikacji są zawarte w jednym obrazie kontenera, eliminując potrzebę instalacji i konfiguracji na docelowych maszynach.
    \item \textbf{Skalowalność:} Łatwo skalowalne aplikacje poprzez uruchamianie wielu instancji kontenerów.
\end{itemize}

\subsubsection{Tworzenie Obrazów Dockera}

Docker wykorzystuje plik \texttt{Dockerfile}, aby definiować obrazy kontenerów, w tym zależności aplikacji, system operacyjny, konfiguracje i pliki aplikacji. Na przykład \texttt{Dockerfile} dla aplikacji Java może zawierać instrukcje do kopiowania plików JAR do obrazu i uruchamiania aplikacji.

Aby zbudować obraz Dockera, można użyć polecenia:

\begin{verbatim}
docker build -t myapp .
\end{verbatim}

Polecenie to tworzy obraz Dockera na podstawie instrukcji zawartych w pliku \texttt{Dockerfile} i pozwala na uruchamianie aplikacji w kontenerze.

\subsection{Integracja Maven i Docker}

Projekt łączy Maven i Docker, aby wykorzystać mocne strony obu narzędzi. Maven zarządza kompilacją aplikacji i zależnościami, podczas gdy Docker zapewnia przenośne, izolowane środowisko wykonawcze. Razem umożliwiają pełną automatyzację procesu budowy i płynne wdrażanie aplikacji w środowiskach deweloperskich i produkcyjnych.

Na przykład Docker może uruchomić kontener z aplikacją wcześniej zbudowaną za pomocą Maven. Docker zapewnia spójne wersje aplikacji i upraszcza wdrażanie w różnych środowiskach.
