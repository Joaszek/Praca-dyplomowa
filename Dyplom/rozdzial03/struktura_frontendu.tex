\section{Struktura katalogów w aplikacji frontendowej}

Struktura katalogów w aplikacji frontendowej jest zorganizowana w sposób, który umożliwia łatwą skalowalność i utrzymanie kodu. Poniżej przedstawiono główną strukturę folderów aplikacji, która jest typowa dla aplikacji React.
{\footnotesize
\begin{verbatim}
frontend
|-- node_modules
|-- public
|-- src
    |-- blockchain
        |-- accounts
        |-- blocks
        |-- transactions
    |-- clientOptions
        |-- predictPrices
        |-- simulateTransaction
    |-- cryptocurrency
        |-- categories
        |-- gainersAndLosers
        |-- globalMarket
        |-- historicalData
        |-- ranking
    |-- login
    |-- mainpage
    |-- resources
        |-- converter
        |-- directory
        |-- news
    |-- signup
    |-- tokens
        |-- collections
        |-- nftStatistics
    |-- App.css
    |-- App.js
    |-- App.test.js
    |-- header.css
    |-- header.js
    |-- index.css
    |-- index.js
    |-- logo.svg
\end{verbatim}
}

\subsection{Opis folderów}

- **\texttt{node\_modules/}**: Zawiera zainstalowane zależności.
- **\texttt{public/}**: Zawiera pliki statyczne.
- **\texttt{src/}**: Główny katalog z kodem źródłowym aplikacji.
  - **\texttt{blockchain/}**: Zawiera pliki do obsługi blockchainów(block, account,transaction). 
	- **\texttt{clientOptions/}**: Zawiera pliki do obsługi przewidywania cen oraz symulowania transakcji. 
	- **\texttt{cryptocurrency/}**: Zawiera pliki do obsługi kryptowalut 
	- **\texttt{login/}**: Zawiera pliki do obsługi logowania. 
	- **\texttt{mainpage/}**: Zawiera pliki do obsługi głównej strony. 
	- **\texttt{resources/}**: Zawiera pliki do obsługi konwersji walut, wiadomości oraz dodatkowych zasobów. 
	- **\texttt{signup/}**: Zawiera pliki do obsługi rejestracji. 
	- **\texttt{tokens/}**: Zawiera pliki do obsługi tokenów. 
	- **\texttt{header.js}**: Jest to plik z nagłówkiem, który jest na każdej stronie. 
  - **\texttt{App.js}**: Główny komponent aplikacji, który pełni rolę kontenera dla pozostałych komponentów. 
  - **\texttt{index.js}**: Punkt wejścia aplikacji. Jest to miejsce, gdzie aplikacja React jest renderowana na stronie, a także, gdzie inicjalizowane są wszystkie globalne ustawienia, takie jak konfiguracja routera.
