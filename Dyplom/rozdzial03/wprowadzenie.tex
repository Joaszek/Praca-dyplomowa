\section{Wprowadzenie}
Rozdział ten opisuje proces implementacji systemu, który został zaprojektowany w celu spełnienia założonych wymagań funkcjonalnych oraz niefunkcjonalnych. Proces implementacji obejmuje wszystkie kluczowe kroki, począwszy od konfiguracji środowiska, przez rozwój aplikacji backendowej i frontendowej, aż po integrację z systemami zewnętrznymi oraz konfigurację kontenerów Dockerowych. Celem tego rozdziału jest przedstawienie szczegółowego opisu poszczególnych etapów tworzenia aplikacji, użytych technologii, narzędzi oraz metod, które umożliwiły realizację założonego projektu.

W niniejszym rozdziale omówione zostaną aspekty techniczne implementacji, w tym struktura repozytoriów kodu, konfiguracja środowiska developerskiego, użyte frameworki i biblioteki, oraz proces budowy i wdrożenia aplikacji. Zostaną również przedstawione główne decyzje architektoniczne, które wpłynęły na projektowanie aplikacji oraz integrację z systemami zewnętrznymi, w tym bazami danych, serwerami aplikacji oraz kontenerami Dockerowymi.

Podczas implementacji, szczególną uwagę poświęcono skalowalności aplikacji, zapewnieniu bezpieczeństwa danych oraz optymalizacji wydajności. Każdy z komponentów systemu – zarówno backend jak i frontend – został zaprojektowany w sposób umożliwiający łatwą rozbudowę oraz integrację z dodatkowymi usługami w przyszłości.