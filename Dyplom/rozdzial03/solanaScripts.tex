\subsection{Folder \texttt{solanaScripts} – działanie, wywoływanie, Dockerfile i biblioteki}

Folder \texttt{solanaScripts} zawiera skrypty odpowiedzialne za interakcję z blockchainem Solany. Skrypty te są wykorzystywane do wykonywania operacji związanych z blockchainem, takich jak tworzenie konta, wykonywanie operacji airdrop oraz symulowanie transakcji.

\subsubsection{Działanie folderu \texttt{solanaScripts}}

Folder \texttt{solanaScripts} zawiera skrypty napisane w języku JavaScript, które są odpowiedzialne za komunikację z blockchainem Solany. Skrypty te wykorzystują bibliotekę \texttt{@solana/web3.js}, która umożliwia interakcję z blockchainem Solany. Skrypty w tym folderze są wywoływane przez backend aplikacji, gdy użytkownik wysyła odpowiednie zapytanie.

Struktura folderu \texttt{solanaScripts} może wyglądać następująco:

\begin{verbatim}
solanaScripts/
|-- airdrop.js  # Skrypt do wykonania operacji airdrop
|-- createTransaction.js  # Skrypt do symulowania transakcji
|-- create.js  # Skrypt do stworzenia konta
\end{verbatim}

\subsubsection{Wywoływanie skryptów z aplikacji}

Skrypty w folderze \texttt{solanaScripts} są wywoływane z aplikacji backendowej w momencie, gdy użytkownik wysyła odpowiednie zapytanie. Aplikacja backendowa  wysyła zapytanie do kontenera Nodejs, który korzystając z biblioteki \texttt{'@solana/web3.js'} uruchamia odpowiednie skrypty Solany.

Przykład na podstawie skryptu \texttt{airdrop}:

\begin{verbatim}
const { Connection, clusterApiUrl, PublicKey, LAMPORTS_PER_SOL } = require('@solana/web3.js');

const connection = new Connection(clusterApiUrl('devnet'), 'confirmed');

async function requestAirdrop(publicKeyBase58) {
    try {
        const publicKey = new PublicKey(publicKeyBase58);

        const airdropSignature = await connection.requestAirdrop(
            publicKey,
            1 * LAMPORTS_PER_SOL
        );

        await connection.confirmTransaction(airdropSignature, 'confirmed');

        console.log(JSON.stringify({
            publicKey: publicKeyBase58,
            message: 'Airdrop of 1 SOL completed',
            transactionSignature: airdropSignature
        }));
    } catch (error) {
        console.error('Airdrop failed:', error);
        console.log(JSON.stringify({
            publicKey: publicKeyBase58,
            message: 'Airdrop failed',
            error: error.message
        }));
    }
}

const publicKeyBase58 = process.argv[2];
if (!publicKeyBase58) {
    console.error('Please provide a public key as an argument.');
    process.exit(1);
}

requestAirdrop(publicKeyBase58);

\end{verbatim}

\subsubsection{Dockerfile – budowanie obrazu dla skryptów Solany}

Skrypty w folderze \texttt{solanaScripts} są częścią aplikacji uruchamianej w kontenerze Docker. Aby zapewnić, że skrypty Solany będą działać w odpowiednim środowisku, został stworzony plik \texttt{Dockerfile}, który buduje odpowiedni obraz Docker.

Plik \texttt{Dockerfile}:

\begin{verbatim}
FROM node:18

WORKDIR /app

COPY package*.json ./

RUN npm install

COPY . .

EXPOSE 3001

CMD ["npm", "start"]
\end{verbatim}

1. \texttt{FROM node:18-slim} – używa obrazu bazowego Node.js w wersji 18, który jest lekki i odpowiedni do uruchamiania skryptów JavaScript.
2. \texttt{WORKDIR /app} – ustawia katalog roboczy w kontenerze na \texttt{/app}.
3. \texttt{COPY package.json ./} – kopiuje pliki konfiguracyjne npm, które zawierają zależności do zainstalowania.
4. \texttt{RUN npm install} – instaluje zależności zapisane w \texttt{package.json}.
5. \texttt{COPY . .} – kopiuje foldery do katalogu roboczego w kontenerze.
6. \texttt{CMD ["npm", "run"]} – uruchamia serwer do słuchania na zapytania.

Kontener jest budowany wraz z kontenerem frontend, backend oraz bazą danych za pomocą komendy:

\begin{verbatim}
docker compose up --build
\end{verbatim}

\subsubsection{Biblioteki używane w skryptach Solany}

Skrypty w folderze \texttt{solanaScripts} korzystają z kilku ważnych bibliotek, które umożliwiają interakcję z blockchainem Solany oraz przetwarzanie danych:

- **\texttt{@solana/web3.js}**: Jest to oficjalna biblioteka JavaScript do interakcji z blockchainem Solany. 
  
  Przykład użycia tej biblioteki do stworzenia konta:
  \begin{verbatim}
const { Keypair } = require('@solana/web3.js');

function createSolanaAccount() {
    const newAccount = Keypair.generate();

    const publicKey = newAccount.publicKey.toBase58();

    const secretKey = Buffer.from(newAccount.secretKey).toString('base64');

    console.log(JSON.stringify({ publicKey, secretKey }));
}

createSolanaAccount();
  \end{verbatim}