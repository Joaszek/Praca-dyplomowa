\section{Podział kodu}

Podczas implementacji systemu, kod został podzielony na dwie główne części: backend oraz frontend. Każda z tych części realizuje określone zadania, a ich współpraca umożliwia pełne funkcjonowanie aplikacji. W poniższym rozdziale przedstawiono szczegóły dotyczące struktury kodu, podziału na funkcje, a także sposób organizacji obu części systemu.

\subsection{Struktura aplikacji}

Struktura aplikacji jest oparta na dwóch głównych komponentach: backendzie i frontendzie. Backend jest odpowiedzialny za logikę aplikacji, zarządzanie danymi oraz komunikację z bazą danych, natomiast frontend odpowiada za interfejs użytkownika oraz interakcje z backendem. Poniżej przedstawiono szczegółowy opis obu części:

\subsubsection{Backend}
Backend aplikacji jest napisany w języku Java z wykorzystaniem frameworku Spring Boot. Struktura kodu backendowego jest zgodna z najlepszymi praktykami w zakresie organizacji projektów opartych na tym frameworku. Kluczowe elementy struktury backendu to:
\begin{itemize}
    \item \texttt{src/main/java}: Główna część aplikacji, zawierająca logikę biznesową i kontrolery.
    \item \texttt{src/main/resources}: Pliki konfiguracyjne, w tym \texttt{application.properties}, które służą do konfiguracji bazy danych i innych usług.
    \item \texttt{src/test/java}: Testy jednostkowe oraz integracyjne, zapewniające poprawność działania aplikacji.
    \item \texttt{pom.xml}: Plik konfiguracyjny Maven, który zarządza zależnościami aplikacji.
    \item \texttt{Dockerfile}: Konfiguracja dla Docker, umożliwiająca tworzenie kontenerów dla aplikacji backendowej.
\end{itemize}

Backend jest podzielony na różne warstwy:
\begin{itemize}
    \item \textbf{Controller}: Warstwa odpowiedzialna za odbiór żądań HTTP i przekazywanie ich do odpowiednich usług.
    \item \textbf{Service}: Warstwa logiki biznesowej, która przetwarza dane i wykonuje operacje na modelach.
    \item \textbf{Repository}: Warstwa odpowiedzialna za komunikację z bazą danych przy użyciu JPA (Java Persistence API).
\end{itemize}

\subsubsection{Frontend}
Frontend aplikacji jest stworzony z użyciem React, popularnego frameworku JavaScript, który umożliwia tworzenie dynamicznych interfejsów użytkownika. Struktura kodu frontendowego obejmuje:
\begin{itemize}
    \item \texttt{src/}: Główna część aplikacji frontendowej, zawierająca komponenty, funkcje oraz logikę.
    \item \texttt{public/}: Pliki statyczne, takie jak HTML, obrazy, czcionki itp.
    \item \texttt{package.json}: Plik konfiguracyjny npm, zarządzający zależnościami aplikacji.
    \item \texttt{Dockerfile.dev}: Plik konfiguracyjny Docker dla środowiska deweloperskiego.
\end{itemize}

Frontend jest podzielony na następujące elementy:
\begin{itemize}
    \item \textbf{Komponenty}: Aplikacja frontendowa jest zbudowana w oparciu o komponenty React, które reprezentują poszczególne elementy interfejsu użytkownika.
    \item \textbf{Logika aplikacji}: Funkcje obsługujące interakcje użytkownika, komunikację z backendem oraz zarządzanie stanem aplikacji.
    \item \textbf{Routing}: Aplikacja frontendowa wykorzystuje React Router do zarządzania nawigacją pomiędzy różnymi widokami aplikacji.
\end{itemize}

\subsection{Podział na funkcje}

W obydwu częściach aplikacji kod został podzielony na funkcje i metody, które realizują określone zadania. Dzięki temu struktura aplikacji jest czytelna, modularna i łatwa do rozbudowy. Poniżej przedstawiono przykłady podziału na funkcje w backendzie oraz frontendzie.

\subsubsection{Backend}
W backendzie każda funkcjonalność została wydzielona do oddzielnych metod w odpowiednich klasach. Przykłady funkcji backendowych:
\begin{itemize}
    \item \texttt{registerUser}: Funkcja tworząca nowego użytkownika w bazie danych.
    \item \texttt{createSolanaAccount}: Funkcja tworząca konto na Solana devnet.
    \item \texttt{getERC20TokenTransfers}: Funkcja pobierająca wszystkie transfery tokenów ERC-20 związane z podanym adresem Ethereum w określonym zakresie bloków (od startBlock do endBlock). Funkcja zwraca listę transakcji w postaci obiektów \texttt{EthereumTransactionDto}.
		\item \texttt{predictPrices}: Funkcja zwracająca przewidywane ceny na następny tydzień dla wybranej kryptowaluty.
\end{itemize}

\subsubsection{Frontend}
W frontendzie kod również jest podzielony na mniejsze funkcje. Przykłady funkcji frontendowych:
\begin{itemize}
    \item \texttt{fetchData}: Funkcja do pobierania danych z API backendu.
    \item \texttt{handleSubmit}: Funkcja obsługująca zdarzenie wysyłania formularza, aby dostać dane ze specyficznego dnia.
    \item \texttt{navigateToDetails}: Funkcja nawigująca do strony szczegółów na podstawie wprowadzonego adresu.
\end{itemize}

\subsection{Integracja Backend-Frontend}

Backend i frontend komunikują się ze sobą za pomocą API. Backend udostępnia RESTful API, które umożliwia frontendowi wykonywanie operacji takich jak tworzenie, pobieranie, aktualizowanie danych. Frontend, korzystając z React, wysyła żądania HTTP do backendu za pomocą axios i renderuje odpowiednie dane na interfejsie użytkownika.

\subsection{Podsumowanie}

Struktura aplikacji została zaprojektowana w sposób modularny, co umożliwia łatwą rozbudowę i modyfikację poszczególnych komponentów. Zarówno backend, jak i frontend zostały podzielone na funkcje, co pozwala na lepszą organizację kodu oraz łatwiejsze testowanie poszczególnych części aplikacji. Integracja pomiędzy frontendem a backendem została zrealizowana przy pomocy API, co zapewnia elastyczność i łatwość w komunikacji między różnymi warstwami systemu.