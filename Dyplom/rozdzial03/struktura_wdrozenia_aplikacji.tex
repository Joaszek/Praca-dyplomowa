\subsection{Struktura wdrożenia aplikacji}

Proces wdrożenia odbywa się za pomocą Docker Compose, co pozwala na uruchomienie wszystkich kontenerów w jednym kroku. Cała aplikacja – frontend, backend, baza danych oraz usługi zewnętrzne, takie jak API blockchain – są uruchamiane jako kontenery Docker, co zapewnia spójność środowiska oraz łatwość w zarządzaniu aplikacją na różnych etapach jej cyklu życia.

\subsubsection{Proces uruchamiania aplikacji}

Wszystkie kontenery aplikacji są uruchamiane poprzez jedną komendę:

\begin{verbatim}
docker-compose up --build
\end{verbatim}

Polecenie to uruchamia proces budowy obrazów (jeśli wcześniej nie zostały zbudowane) i uruchamia wszystkie kontenery zdefiniowane w pliku \texttt{docker-compose.yml}.

\subsubsection{Baza danych}

W pliku \texttt{docker-compose.yml} kontener bazy danych jest skonfigurowany z użyciem opcji \texttt{external: true}. Oznacza to, że baza danych nie jest tworzona bezpośrednio w kontenerze, lecz jest połączona z zewnętrzną bazą danych. W tym przypadku dane w bazie danych są zapisane i ładowane z określonego źródła, które jest dostępne na zewnątrz aplikacji.

Konfiguracja bazy danych w pliku \texttt{docker-compose.yml} wygląda następująco:

\begin{verbatim}
version: '3.8'

services:
  postgres:
    image: postgres:16
    container_name: postgres_db
    environment:
      POSTGRES_USER: myuser
      POSTGRES_PASSWORD: mypassword
      POSTGRES_DB: mydatabase
    volumes:
      - backend_pg_data:/var/lib/postgresql/data
    ports:
      - "5432:5432"
    networks:
      - my_network

  pgadmin:
    image: dpage/pgadmin4
    container_name: pgadmin
    environment:
      PGADMIN_DEFAULT_EMAIL: admin@admin.com
      PGADMIN_DEFAULT_PASSWORD: admin
    ports:
      - "5050:80"
    depends_on:
      - postgres
    networks:
      - my_network
volumes:
  backend_pg_data:
    external: true
\end{verbatim}

W tej konfiguracji:
- \texttt{postgres\_db} jest połączony z zewnętrzną bazą danych.
- \texttt{external: true} oznacza, że baza danych nie jest tworzona w kontenerze, lecz jest połączona z istniejącym zasobem zewnętrznym.
- Backend jest odpowiedzialny za komunikację z tą bazą danych, zdefiniowaną w środowisku aplikacji.

\subsubsection{Usługi blockchain}

Dane blockchain są pobierane na żądanie za pomocą różnych API, które są wywoływane, gdy aplikacja backendowa otrzyma odpowiedni request. Kiedy backend odbiera zapytanie od frontendu, na przykład o szczegóły transakcji lub saldo konta, backend kontaktuje się z odpowiednim API blockchain (np. API Ethereum, Bitcoin, Solana) w celu pobrania danych.

Backend wykorzystyje biblioteki takie jak \texttt{RestTemplate} do komunikacji z API blockchain. Dane są następnie przetwarzane i zwracane do frontendu, który prezentuje je użytkownikowi.

\begin{verbatim}
private final RestTemplate restTemplate;

    public BitcoinAccountService(@Qualifier("restTemplate") RestTemplate restTemplate) {
        this.restTemplate = restTemplate;
    }

    public BitcoinAccountDto getAllAccountData(String address) {
        String url = "https://api.blockcypher.com/v1/btc/main/addrs/" + address + "/full";

        Optional<BitcoinAccount> bitcoinAccountOptional = Optional.ofNullable(restTemplate.getForObject(url, BitcoinAccount.class));

        if (bitcoinAccountOptional.isPresent()) {
            return BitcoinAccountMapper.mapAccountToAccountDto(bitcoinAccountOptional.get());
        } else {
            return null;
        }
    }
\end{verbatim}

\subsubsection{Komunikacja między frontendem, backendem i blockchain}

Wszystkie te komponenty (frontend, backend, baza danych, API blockchain) komunikują się ze sobą w ramach sieci Docker, która jest zdefiniowana w pliku \texttt{docker-compose.yml}. Dzięki temu, backend ma dostęp do bazy danych PostgreSQL, a także może łączyć się z usługami blockchain, które są dostępne w internecie lub w innych kontenerach, jeśli wymagają one wewnętrznej komunikacji.