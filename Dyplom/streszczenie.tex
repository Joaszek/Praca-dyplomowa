\pdfbookmark[0]{Streszczenie}{streszczenie.1}
\phantomsection
\addcontentsline{toc}{chapter}{Streszczenie}
%% Poniższe zostało niewykorzystane (tj. zrezygnowano z utworzenia nienumerowanego rozdziału na abstrakt)
%%\begingroup
%%\setlength\beforechapskip{48pt} % z jakiegoś powodu była maleńka różnica w położeniu nagłówka rozdziału numerowanego i nienumerowanego
%%\chapter*{\centering Abstrakt}
%%\endgroup
%%\label{sec:abstrakt}
%%Lorem ipsum dolor sit amet eleifend et, congue arcu. Morbi tellus sit amet, massa. Vivamus est id risus. Sed sit amet, libero. Aenean ac ipsum. Mauris vel lectus. 
%%
%%Nam id nulla a adipiscing tortor, dictum ut, lobortis urna. Donec non dui. Cras tempus orci ipsum, molestie quis, lacinia varius nunc, rhoncus purus, consectetuer congue risus. 
\mbox{}\vspace{2cm} % można przesunąć, w zależności od długości streszczenia
\begin{abstract}
\begin{flushleft}

\end{flushleft}
\section*{Streszczenie} Niniejsza praca inżynierska dotyczy zaprojektowania i stworzenia platformy internetowej przeznaczonej dla użytkowników kryptowalut i technologii blockchain. Celem projektu jest nie tylko dostarczenie wiedzy na temat różnych blockchainów, takich jak Bitcoin, Ethereum i Solana, ale także umożliwienie użytkownikom bezpośredniego dostępu do danych tych sieci, w tym przeglądania transakcji, stanu kont oraz monitorowania parametrów sieci. W odróżnieniu od istniejących narzędzi, takich jak Etherscan czy Solscan, platforma oferuje integrację z wieloma blockchainami, co zapewnia użytkownikom kompleksowe informacje i funkcje w jednym miejscu.

\begin{enumerate} \item \textbf{Sprawdzanie danych z różnych blockchainów}: Platforma umożliwia użytkownikom przeglądanie danych z takich sieci jak Bitcoin, Ethereum oraz Solana, w tym śledzenie transakcji, sprawdzanie stanu kont i monitorowanie parametrów sieci. \item \textbf{Integracja z API, np. Alchemy}: Dzięki tej integracji użytkownicy mogą wykonywać zapytania do blockchainów w czasie rzeczywistym, uzyskując dostęp do bieżących danych. \item \textbf{Przewidywanie cen kryptowalut}: Zaimplementowane modele sztucznej inteligencji wspierają użytkowników w podejmowaniu decyzji inwestycyjnych, bazując na analizie danych z blockchainów. \end{enumerate}

Projekt został zrealizowany z użyciem nowoczesnych technologii:

\begin{enumerate} \item \textbf{Frontend}: zbudowany w React. \item \textbf{Backend}: oparty na Spring Boot. \item \textbf{Wdrożenie}: na platformie chmurowej Amazon Web Services (AWS). \item \textbf{Bezpieczeństwo}: Platforma wykorzystuje szyfrowanie danych, aby zapewnić bezpieczne przetwarzanie informacji. \end{enumerate}

Praca obejmuje:

\begin{enumerate} \item \textbf{Analizę wymagań}. \item \textbf{Implementację funkcji edukacyjnych i transakcyjnych}. \item \textbf{Testy funkcjonalne}. \item \textbf{Ocenę stabilności systemu}. \end{enumerate}

W końcowej części omówiono potencjalne kierunki rozwoju platformy, takie jak wsparcie dla dodatkowych blockchainów i rozszerzenie funkcji związanych z analizą danych. Projekt stanowi innowacyjne narzędzie dla użytkowników, umożliwiając im zarówno zdobywanie wiedzy, jak i interakcję z rzeczywistymi danymi blockchainowymi w jednym, zintegrowanym środowisku.

\section*{Summary} This engineering thesis focuses on designing and developing a web platform aimed at users of cryptocurrencies and blockchain technology. The goal of the project is not only to provide knowledge about different blockchains, such as Bitcoin, Ethereum, and Solana, but also to enable users to directly access data from these networks, including transaction tracking, account status, and network parameters monitoring. Unlike existing tools like Etherscan or Solscan, the platform offers integration with multiple blockchains, providing users with comprehensive information and functionalities in a single interface.

\begin{enumerate} \item \textbf{Data checking from multiple blockchains}: The platform allows users to view data from networks such as Bitcoin, Ethereum, and Solana, including transaction tracking, checking account balances, and monitoring network parameters. \item \textbf{Integration with APIs, such as Alchemy}: This integration enables users to query blockchains in real-time, gaining access to up-to-date data. \item \textbf{Cryptocurrency price prediction}: Implemented artificial intelligence models support users in making investment decisions based on blockchain data analysis. \end{enumerate}

The project was developed using modern technologies:

\begin{enumerate} \item \textbf{Frontend}: built with React. \item \textbf{Backend}: utilizes Spring Boot. \item \textbf{Deployment}: on the Amazon Web Services (AWS) cloud. \item \textbf{Security}: The platform employs data encryption to guarantee the secure processing of information. \end{enumerate}

The work includes:

\begin{enumerate} \item \textbf{Requirements analysis}. \item \textbf{Implementation of educational and transactional functions}. \item \textbf{Functional testing}. \item \textbf{Evaluation of system stability}. \end{enumerate}

The final section discusses potential future development directions for the platform, such as support for additional blockchains and expansion of data analysis features. The project represents an innovative tool for users, allowing them to both acquire knowledge and interact with real blockchain data within an integrated environment.
\mykeywords
}
