\pdfbookmark[0]{Streszczenie}{streszczenie.1}
%\phantomsection
%\addcontentsline{toc}{chapter}{Streszczenie}
%%% Poniższe zostało niewykorzystane (tj. zrezygnowano z utworzenia nienumerowanego rozdziału na abstrakt)
%%%\begingroup
%%%\setlength\beforechapskip{48pt} % z jakiegoś powodu była maleńka różnica w położeniu nagłówka rozdziału numerowanego i nienumerowanego
%%%\chapter*{\centering Abstrakt}
%%%\endgroup
%%%\label{sec:abstrakt}
%%%Lorem ipsum dolor sit amet eleifend et, congue arcu. Morbi tellus sit amet, massa. Vivamus est id risus. Sed sit amet, libero. Aenean ac ipsum. Mauris vel lectus. 
%%%
%%%Nam id nulla a adipiscing tortor, dictum ut, lobortis urna. Donec non dui. Cras tempus orci ipsum, molestie quis, lacinia varius nunc, rhoncus purus, consectetuer congue risus. 
%\mbox{}\vspace{2cm} % można przesunąć, w zależności od długości streszczenia
\begin{abstract}
% TO DO: abstrakt powinien być na pół strony - wymaganie to było sprecyzowane w wytycznych !!!!!
W pracy opisano projekt autorskiej platformy internetowej przeznaczonej dla użytkowników kryptowalut i technologii blockchain. Poza dostarczeniem wiedzy na temat różnych blockchainów, takich jak Bitcoin, Ethereum i Solana, osiągniętym rezultatem jest narzędzie programowe, które pozwala na bezpośredni dostęp do danych tych sieci, w tym na przeglądanie transakcji, stanu kont oraz monitorowanie parametrów sieci. W odróżnieniu od innych narzędzi tego typu, jak Etherscan czy Solscan, oferowane narzędzie współpracuje z wieloma blockchainami, dostarczając kompleksowe informacje i funkcje w jednym miejscu. 

% TO DO: dodać coś o analizie danych z wykorzystaniem AI - wszak to jest w tytule pracy
%

Projekt zrealizowano z użyciem nowoczesnych technologii: React (frontend), SpringBoot (backend), Amazon Web Services (AWS) (wdrożenie) oraz wybranych bibliotek szyfrowania. 
% TO DO: Można dodac zdanie (jeśli znajdzie się miejsce): W podsumowaniu określono potencjalne kierunki rozwoju projektu, obejmujące obsługę dodatkowych blockchainów i rozbudowę funkcji analizy danych. 

\end{abstract}
\mykeywords

{
\selectlanguage{english}
\begin{abstract}
The paper describes the project of an original web platform intended for users of cryptocurrencies and blockchain technology. In addition to providing knowledge about various blockchains, such as Bitcoin, Ethereum, and Solana, the result achieved is a software tool that offers direct access to data from these networks, including viewing transactions, account status, and monitoring network parameters. Unlike other tools of this type, such as Etherscan or Solscan, the offered tool works with many blockchains, providing comprehensive information and functions in one place.

% TO DO: add something about data analysis using AI - after all, it is in the title of the paper
%
The project was implemented using modern technologies: React (frontend), SpringBoot (backend), Amazon Web Services (AWS) (implementation) and selected encryption libraries.
\end{abstract}
\mykeywords
}