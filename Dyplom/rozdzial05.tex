\chapter{Wdrożenie}
\section{Wstęp}
Wdrożenie aplikacji jest ostatnim etapem procesu tworzenia oprogramowania. Proces ten polega na przygotowaniu odpowiedniej konfiguracji środowiska i infrastruktury pod kątem skalowalności, bezpieczeństwa i ochrony przed awarią serwera. W projekcie wykorzystano technologię Docker oraz AWS, jednak ze względu na koszty AWS nie będzie wykorzystywany w dłuższym okresie czasu.

\section{Lokalne środowisko}
% TO DO: sekcja nie może składać się z jednego akapitu. Sekcja jest fragmentem rozdziału, musi więc w sobie nieść nieco więcej treści. 
% Proszę więc albo rozwinąć nieco bardziej opis, albo zamienić sekcję na coś innego (paragraf, wyliczenie czy inny blok tekstu - odpowiednio do potrzeb)
\subsection{Pobranie źródeł projektu}
Pierwszym krokiem do wdrożenia aplikacji jest pobranie źródeł projekt. Aby to zrobić należy wpisać w terminal:
\begin{lstlisting}[basicstyle=\ttfamily\footnotesite]
git clone --recurse-submodules https://github.com/Bachelor-Thesis-Szewior-Joachim/deployment.git
\end{lstlisting}
Polecenie to utworzy lokalny klon zdalnego repozytorium kodu. Aby dało się to polecenie uruchomić, na komputerze użytkownika musi być zaistalowane narzędzie git.
Po pobarni źródeł można przejść do następnych kroków, tj.\ do zbudowania i uruchomienia odpowiednich dockerowych obrazów.

\subsection{Budowanie obrazów}
Aby uruchomić aplikację, należy zbudować obrazy dla 4 kontenerów:
\begin{itemize}
    \item \texttt{frontend}
    \item \texttt{backend}
    \item \texttt{solana\_scripts}
    \item \texttt{model\_server}
\end{itemize}

Obrazy można także pobrać z Docker Hub:
\begin{lstlisting}[basicstyle=\footnotesize\ttfamily]
docker pull joaszek/bachelor-thesis:model_server_image
docker pull joaszek/bachelor-thesis:solana_scripts_image
docker pull joaszek/bachelor-thesis:backend_image
docker pull joaszek/bachelor-thesis:frontend_image
docker pull joaszek/bachelor-thesis:postgres_image
\end{lstlisting}

\begin{lstlisting}[basicstyle=\footnotesize\ttfamily,tabsize=2]
version: "3.8"

services:
  postgres:
    image: postgres:16
    container_name: postgres_db
    environment:
      POSTGRES_USER: myuser
      POSTGRES_PASSWORD: mypassword
      POSTGRES_DB: mydatabase
    volumes:
      - ./backup.sql:/docker-entrypoint-initdb.d/backup.sql
      - backend_pg_data:/var/lib/postgresql/data
    ports:
      - "5432:5432"
    networks:
      - my_network

  pgadmin:
    image: dpage/pgadmin4
    container_name: pgadmin
    environment:
      PGADMIN_DEFAULT_EMAIL: admin@admin.com
      PGADMIN_DEFAULT_PASSWORD: admin
    ports:
      - "5050:80"
    depends_on:
      - postgres
    networks:
      - my_network

  backend_app:
    image: backend_image
    container_name: spring_app
    privileged: true
    volumes:
      - ./backend:/app
      - /var/run/docker.sock:/var/run/docker.sock
    environment:
      SPRING_DATASOURCE_URL: jdbc:postgresql://postgres:5432/mydatabase
      SPRING_DATASOURCE_USERNAME: myuser
      SPRING_DATASOURCE_PASSWORD: mypassword
      SPRING_JPA_HIBERNATE_DDL_AUTO: update
      SPRING_JPA_SHOW_SQL: "true"
      SPRING_JPA_PROPERTIES_HIBERNATE_DIALECT: org.hibernate.dialect.PostgreSQLDialect
      COINMARKETCAP_API_URL: d42f0690-3288-4f73-8230-da9ac5135859
    ports:
      - "8080:8080"
    depends_on:
      - postgres
    networks:
      - my_network

  solana_scripts_service:
    image: solana_scripts_image
    container_name: solana_scripts_service
    volumes:
      - ./backend/src/main/java/org/example/backend/client/client/service/solanaScripts:/app
    working_dir: /app
    ports:
      - "3001:3001"
    command: ["npm", "start"]
    networks:
      - my_network

  model_server:
    image: model_server
    container_name: model_server
    ports:
      - "5000:5000"
    networks:
      - my_network
    volumes:
      - ./models:/models

  frontend_app:
    image: frontend_image
    container_name: frontend_app
    volumes:
      - ./frontend:/usr/app
    ports:
      - "3000:3000"
    command: ["npm", "start"]
    networks:
      - my_network

volumes:
  backend_pg_data:

networks:
  my_network:
    driver: bridge
\end{lstlisting}

Następnie należy uruchomić komendę \texttt{docker compose up} w folderze, w którym jest \texttt{docker-compose.yml} aby postawić aplikację na swoim komputerze.

\section{Środowisko AWS}
\subsection{Serwisy}
Aplikacja została wdrożona w środowisku AWS z wykorzystaniem usługi ECS (Elastic Container Service) w trybie Fargate. Skonfigurowano następujące serwisy:
\begin{itemize}
    \item \textbf{Postgres}:
    \begin{itemize}
        \item Obraz: \url{503561436372.dkr.ecr.eu-north-1.amazonaws.com/bachelor/deployment:postgres_custom}
        \item Port: \texttt{5432}
        \item Zmienne środowiskowe:
        \begin{itemize}
            \item \texttt{POSTGRES\_USER: myuser}
            \item \texttt{POSTGRES\_PASSWORD: mypassword}
            \item \texttt{POSTGRES\_DB: mydatabase}
        \end{itemize}
    \end{itemize}
    \item \textbf{PgAdmin}:
    \begin{itemize}
        \item Obraz: \texttt{dpage/pgadmin4}
        \item Port: \texttt{5050}
        \item Zmienne środowiskowe:
        \begin{itemize}
            \item \texttt{PGADMIN\_DEFAULT\_EMAIL: admin@admin.com}
            \item \texttt{PGADMIN\_DEFAULT\_PASSWORD: admin}
        \end{itemize}
    \end{itemize}
    \item \textbf{Backend App}:
    \begin{itemize}
        \item Obraz: \url{503561436372.dkr.ecr.eu-north-1.amazonaws.com/bachelor/deployment:backend_image_latest}
        \item Port: \texttt{8080}
        \item Zmienne środowiskowe:
        \begin{itemize}
            \item \texttt{SPRING\_JPA\_SHOW\_SQL: true}
            \item \texttt{SPRING\_DATASOURCE\_URL: jdbc:postgresql://postgres:5432/mydatabase}
            \item \texttt{SPRING\_DATASOURCE\_PASSWORD: mypassword}
            \item \texttt{SPRING\_JPA\_HIBERNATE\_DDL\_AUTO: update}
            \item \texttt{SPRING\_DATASOURCE\_USERNAME: myuser}
            \item \texttt{COINMARKETCAP\_API\_URL: ...}
        \end{itemize}
    \end{itemize}
    \item \textbf{Solana Scripts Service}:
    \begin{itemize}
        \item Obraz: \url{503561436372.dkr.ecr.eu-north-1.amazonaws.com/bachelor/deployment:solana_scripts_image_latest}
        \item Port: \texttt{3001}
        \item Polecenie startowe: \texttt{npm start}
    \end{itemize}
    \item \textbf{Model Server}:
    \begin{itemize}
        \item Obraz: \url{503561436372.dkr.ecr.eu-north-1.amazonaws.com/bachelor/deployment:model_server_image_latest}
        \item Port: \texttt{5000}
    \end{itemize}
    \item \textbf{Frontend App}:
    \begin{itemize}
        \item Obraz: \url{503561436372.dkr.ecr.eu-north-1.amazonaws.com/bachelor/deployment:frontend_image_latest}
        \item Port: \texttt{3000}
        \item Polecenie startowe: \texttt{npm start}
    \end{itemize}
\end{itemize}

\subsection{Skalowanie i infrastruktura}
W ramach infrastruktury AWS zostały skonfigurowane:
\begin{itemize}
    \item Dwa subnety:
    \begin{itemize}
        \item \textbf{Publiczny subnet} do komunikacji z internetem.
        \item \textbf{Prywatny subnet} dla wewnętrznych usług, takich jak baza danych.
    \end{itemize}
    \item Specjalna sieć VPC (\textit{Virtual Private Cloud}) dedykowana dla tej aplikacji.
    \item Security Group z odpowiednimi regułami dostępu, zezwalającymi na ruch przychodzący i wychodzący na portach używanych przez aplikację (3000, 8080, 5432, 5050, 5000).
    \item Rola IAM o nazwie \texttt{ECS\_Role}, która umożliwia ECS dostęp do obrazów w ECR i logowanie do CloudWatch.
    \item Load Balancer, który rozdziela ruch do aplikacji i zapewnia wysoką dostępność.
\end{itemize}
