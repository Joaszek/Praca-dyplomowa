\chapter{Podsumowanie}
Tematem pracy inżynierskiej było zaprojektowanie, implementacja oraz wdrożenie platformy informacyjno-transakcyjnej dla użytkowników kryptowalut, integrującej technologie blockchain oraz sztuczną inteligencję do analizy i predykcji cen kryptowalut. Głównym jej zaś celem było stworzenie rozwiązania, które łączy funkcje dostępne w narzędziach takich jak Etherscan czy Solscan, oraz oferuje wsparcie dla wielu blockchainów, w tym Bitcoin, Ethereum i Solana. 

Cel pracy udało się zrealizować. Platforma zbudowana w ramach pracy pozwala, m.in., na przeglądanie danych transakcyjnych, obserwację stanu konta i parametrów sieci blockchain. Umożliwia także prognozowanie cen kryptowalut za pomocą modeli AI oraz symuluje transakcję w sieci Solana. Bezpieczeństwo przetwarzania danych blockchain odgrywa w tym projekcie ważną rolę i obejmuje mechanizmy szyfrowania i autoryzacji według tokenów JWT.

Jednym z pierwotnych założeń projektu było stworzenie własnego blockchaina. Niestety, z~powodu ograniczonego czasu nie udało się zrealizować tego celu. Mimo tego, platforma została zaprojektowana z uwzględnieniem wsparcia dla popularnych blockchainów, takich jak Bitcoin, Ethereum i Solana, które dostarczają wystarczająco dużą bazę do analizy i predykcji. Dodatkowo, w związku z ograniczeniami budżetowymi, nie udało się zapewnić stałej aktualizacji danych z zewnętrznych baz danych blockchainowych. API, które umożliwia bieżące pobieranie danych o kryptowalutach i ich historii jest płatne, co uniemożliwiło jego pełne wykorzystanie w~systemie. W efekcie dane w bazie są statyczne i nie są aktualizowane w czasie rzeczywistym.

Pod względem technicznym projekt opiera się na architekturze wielowarstwowej, gdzie frontend zbudowano w technologii React, backend w Spring Boot, a integracje z zewnętrznymi API, takimi jak Alchemy czy CoinMarketCap, wspierają przetwarzanie danych. Modele predykcyjne opierają się na technologii LSTM, natomiast skrypty związane z blockchainem Solana wykorzystują bibliotekę \texttt{@solana/web3.js}. Wszystkie komponenty zostały skonteneryzowane przy użyciu Dockera, co zapewnia łatwość wdrożenia, izolację środowisk i skalowalność. 

Z uwagi na brak dostępu do pełnych i aktualnych danych z rynku kryptowalut, model predykcyjny, który miał przewidywać ceny kryptowalut, nie jest w stanie aktualizować się. W dalszej perspektywie może więc dawać gorsze wyniki, ponieważ nie był odpowiednio wytrenowany na danych rzeczywistych i nie będzie w stanie uwzględniać bieżących trendów rynkowych.

W trakcie testowania platformy dane do testów wygenerowano sztucznie, przez zmockowanie, a nie wykorzystanie rzeczywistych danych pochodzących z blockchainów. Choć takie dane umożliwiły przeprowadzenie podstawowych testów funkcjonalnych, nie oddają one w pełni zachowań systemu w warunkach rzeczywistego użytkowania. 

Platforma zapewnia użytkownikom interfejs do analizowania danych blockchain, zarządzania NFT i monitorowania wskaźników rynkowych, takich jak „indeks strachu i chciwości”. Wymagania pozafunkcjonalne systemu to wydajność, skalowalność i odporność na ataki, co czyni platformę bezpiecznym i niezawodnym narzędziem dla entuzjastów blockchain. Wdrożenie odbywa się w chmurze Amazon Web Services (AWS).
