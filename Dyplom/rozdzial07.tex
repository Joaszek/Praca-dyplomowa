\chapter{Podsumowanie}
Tematem pracy inżynierskiej było zaprojektowanie, implementacja oraz wdrożenie platformy informacyjno-transakcyjnej dla użytkowników kryptowalut, integrującej technologie blockchain oraz sztuczną inteligencję do analizy i predykcji cen kryptowalut. Celem projektu było stworzenie rozwiązania, które łączy funkcjonalności dostępne w narzędziach takich jak Etherscan czy Solscan, jednocześnie oferując wsparcie dla wielu blockchainów, w tym Bitcoin, Ethereum i Solana.

Platforma realizuje m.in. przeglądanie danych transakcyjnych, obserwację stanu konta i parametrów sieci blockchain. Umożliwia także prognozowanie cen kryptowalut za pomocą modeli AI oraz symuluje transakcję w sieci Solana. Bezpieczeństwo przetwarzania danych blockchain odgrywa w tym projekcie ważną rolę i obejmuje mechanizmy szyfrowania i autoryzacji według tokenów JWT.

Pod względem technicznym projekt opiera się na architekturze wielowarstwowej, gdzie frontend zbudowano w technologii React, backend w Spring Boot, a integracje z zewnętrznymi API, takimi jak Alchemy czy CoinMarketCap, wspierają przetwarzanie danych. Modele predykcyjne opierają się na technologii LSTM, natomiast skrypty związane z blockchainem Solana wykorzystują bibliotekę \texttt{@solana/web3.js}. Wszystkie komponenty zostały skonteneryzowane przy użyciu Dockera, co zapewnia łatwość wdrożenia, izolację środowisk i skalowalność. Wdrożenie odbywa się w chmurze Amazon Web Services (AWS), co zwiększa niezawodność i elastyczność systemu.

Platforma zapewnia użytkownikom interfejs do analizowania danych blockchain, zarządzania NFT i monitorowania wskaźników rynkowych, takich jak „indeks strachu i chciwości”. Wymagania pozafunkcjonalne systemu to wydajność, skalowalność i odporność na ataki, co czyni platformę bezpiecznym i niezawodnym narzędziem dla entuzjastów blockchain.