\chapter{Analiza wymagań}
% TO DO: poniższa sekcja została przeniesiona z pierwszego rozdziału, bo na analizę wymagań mamy specjalny rozdział (we wstępie pisze się zwykle o celu i ogólnych założeniach, szczegóły omawia się później).

\section{Zakres funkcjonalny platformy}

Platforma będzie oferować użytkownikom szeroki zakres funkcji, umożliwiając im nie tylko naukę, ale także interakcję z rzeczywistymi danymi blockchainowymi:

\begin{enumerate} \item \textbf{Sprawdzanie danych z blockchainów}: użytkownicy będą mogli przeglądać aktualne informacje o transakcjach, stanie kont i opłatach w wybranych blockchainach. Dzięki integracji z Alchemy, platforma będzie zapewniała bezpośredni dostęp do danych blockchainowych w czasie rzeczywistym. \item \textbf{Symulacja transakcji na blockchainie Solana}: platforma umożliwi użytkownikom przeprowadzenie symulowanych transakcji, co pozwoli na lepsze zrozumienie mechanizmów przesyłania środków bez ryzyka utraty realnych środków. \item \textbf{Przewidywanie cen kryptowalut}: z wykorzystaniem modeli sztucznej inteligencji użytkownicy będą mogli uzyskać prognozy dotyczące cen kryptowalut, co pomoże im w podejmowaniu decyzji inwestycyjnych. \item \textbf{Dostęp do materiałów edukacyjnych}: platforma będzie zawierać linki do kursów wideo, artykułów oraz innych materiałów edukacyjnych, które pozwolą użytkownikom poszerzać wiedzę na temat technologii blockchain. \item \textbf{Konwersja walut i kryptowalut}: funkcja konwertera ułatwi przeliczenia pomiędzy walutami tradycyjnymi a kryptowalutami, co jest istotne dla inwestorów. \end{enumerate}

Platforma wyróżniać się będzie możliwością \textbf{wyboru konkretnego blockchaina do analiz i operacji}, co odróżnia ją od istniejących serwisów, takich jak Etherscan (skoncentrowanego na Ethereum) czy Solscan (dedykowanego Solanie). Dzięki integracji z API użytkownicy będą mieli dostęp do bieżących danych o transakcjach i stanie blockchainów, co zwiększy wartość informacyjną platformy i pozwoli na podejmowanie bardziej świadomych decyzji inwestycyjnych.


\section{Architektura systemu} 




\subsection{Komponenty systemu}
% TO DO: przypominam o regułach związanych z kropkami na końcach zdań i przy skrótach - mówiłem o tym na spotkaniu on-line!!!

% TO DO: czy numeracja coś tutaj wnosi? Jeśli nie, to jest niepotrzebna. O numeracji wspominałem, ale przy wyliczaniu funkcji systemu
%\begin{enumerate}[label=\arabic*.]
\subsection{Interakcje między komponentami}


% TO DO: czy numeracja coś tutaj wnosi? Jeśli nie, to jest niepotrzebna



\section{Wymagania aplikacji} %DONE: aktualizaja opisu rozmowy na ostanim spotkaniu



\subsection{Przykłady użycia aplikacji}


\subsection{Wymagania niefunkcjonalne}


\section{Podsumowanie}

